%%%%%%%%%%%%%%%%%%%%%%%%%%%%%%%%%%%%%%%%%
% Programming/Coding Assignment
% LaTeX Template
%
% This template has been downloaded from:
% http://www.latextemplates.com
%
% Original author:
% Ted Pavlic (http://www.tedpavlic.com)
%
% Note:
% The \lipsum[#] commands throughout this template generate dummy text
% to fill the template out. These commands should all be removed when 
% writing assignment content.
%
% This template uses a Perl script as an example snippet of code, most other
% languages are also usable. Configure them in the "CODE INCLUSION 
% CONFIGURATION" section.
%
%%%%%%%%%%%%%%%%%%%%%%%%%%%%%%%%%%%%%%%%%

%----------------------------------------------------------------------------------------
%	PACKAGES AND OTHER DOCUMENT CONFIGURATIONS
%----------------------------------------------------------------------------------------

\documentclass{article}

\usepackage{fancyhdr} % Required for custom headers
\usepackage{lastpage} % Required to determine the last page for the footer
\usepackage{extramarks} % Required for headers and footers
\usepackage[usenames,dvipsnames]{color} % Required for custom colors
\usepackage{graphicx} % Required to insert images
\usepackage{listings} % Required for insertion of code
\usepackage{courier} % Required for the courier font
\usepackage{lipsum} % Used for inserting dummy 'Lorem ipsum' text into the template
\usepackage{amsmath,amsfonts,amsthm}

% Margins
\topmargin=-0.45in
\evensidemargin=0in
\oddsidemargin=0in
\textwidth=6.5in
\textheight=9.0in
\headsep=0.25in

\linespread{1.1} % Line spacing

% Set up the header and footer
\pagestyle{fancy}
\lhead{\hmwkAuthorName} % Top left header
%\chead{\hmwkClass:\ \hmwkTitle} % Top center head
\rhead{\firstxmark} % Top right header
\lfoot{\lastxmark} % Bottom left footer
\cfoot{} % Bottom center footer
\rfoot{Page\ \thepage\ of\ \protect\pageref{LastPage}} % Bottom right footer
\renewcommand\headrulewidth{0.4pt} % Size of the header rule
\renewcommand\footrulewidth{0.4pt} % Size of the footer rule

\setlength\parindent{0pt} % Removes all indentation from paragraphs

%----------------------------------------------------------------------------------------
%	CODE INCLUSION CONFIGURATION
%----------------------------------------------------------------------------------------


%%PEARL
\definecolor{MyDarkGreen}{rgb}{0.0,0.4,0.0} % This is the color used for comments
\lstloadlanguages{Perl} % Load Perl syntax for listings, for a list of other languages supported see: ftp://ftp.tex.ac.uk/tex-archive/macros/latex/contrib/listings/listings.pdf
\lstset{language=Perl, % Use Perl in this example
        frame=single, % Single frame around code
        basicstyle=\small\ttfamily, % Use small true type font
        keywordstyle=[1]\color{Blue}\bf, % Perl functions bold and blue
        keywordstyle=[2]\color{Purple}, % Perl function arguments purple
        keywordstyle=[3]\color{Blue}\underbar, % Custom functions underlined and blue
        identifierstyle=, % Nothing special about identifiers                                         
        commentstyle=\usefont{T1}{pcr}{m}{sl}\color{MyDarkGreen}\small, % Comments small dark green courier font
        stringstyle=\color{Purple}, % Strings are purple
        showstringspaces=false, % Don't put marks in string spaces
        tabsize=5, % 5 spaces per tab
        %
        % Put standard Perl functions not included in the default language here
        morekeywords={rand},
        %
        % Put Perl function parameters here
        morekeywords=[2]{on, off, interp},
        %
        % Put user defined functions here
        morekeywords=[3]{test},
       	%
        morecomment=[l][\color{Blue}]{...}, % Line continuation (...) like blue comment
        numbers=left, % Line numbers on left
        firstnumber=1, % Line numbers start with line 1
        numberstyle=\tiny\color{Blue}, % Line numbers are blue and small
        stepnumber=5 % Line numbers go in steps of 5
}

% Creates a new command to include a perl script, the first parameter is the filename of the script (without .pl), the second parameter is the caption
\newcommand{\perlscript}[2]{
\begin{itemize}
\item[]\lstinputlisting[caption=#2,label=#1]{#1.pl}
\end{itemize}
}

%%C
\definecolor{MyDarkGreen}{rgb}{0.0,0.4,0.0} % This is the color used for comments
\lstloadlanguages{[ISO]C++} 
\lstset{language=C++, % Use C in this example
        frame=single, % Single frame around code
        basicstyle=\small\ttfamily, % Use small true type font
        keywordstyle=[1]\color{Blue}\bf, % Perl functions bold and blue
        keywordstyle=[2]\color{Purple}, % Perl function arguments purple
        keywordstyle=[3]\color{Blue}\underbar, % Custom functions underlined and blue
        identifierstyle=, % Nothing special about identifiers                                         
        commentstyle=\usefont{T1}{pcr}{m}{sl}\color{MyDarkGreen}\small, % Comments small dark green courier font
        stringstyle=\color{Purple}, % Strings are purple
        showstringspaces=false, % Don't put marks in string spaces
        tabsize=5, % 5 spaces per tab
        %
        % Put standard Perl functions not included in the default language here
        morekeywords={rand},
        %
        % Put Perl function parameters here
        morekeywords=[2]{on, off, interp},
        %
        % Put user defined functions here
        morekeywords=[3]{test},
       	%
        morecomment=[l][\color{Blue}]{...}, % Line continuation (...) like blue comment
        numbers=left, % Line numbers on left
        firstnumber=1, % Line numbers start with line 1
        numberstyle=\tiny\color{Blue}, % Line numbers are blue and small
        stepnumber=5 % Line numbers go in steps of 5
}

% Creates a new command to include a perl script, the first parameter is the filename of the script (without .cpp), the second parameter is the caption
\newcommand{\cscript}[2]{
\begin{itemize}
\item[]\lstinputlisting[caption=#2,label=#1]{#1.cpp}
\end{itemize}
}


% Creates a new command to include a perl script, the first parameter is the filename of the script (without .py), the second parameter is the caption
\newcommand{\pyscript}[2]{
\begin{itemize}
\item[]\lstinputlisting[caption=#2,label=#1]{#1.py}
\end{itemize}
}


% Creates a new command to include a perl script, the first parameter is the filename of the script (without .pov, the second parameter is the caption
\newcommand{\povscript}[2]{
\begin{itemize}
\item[]\lstinputlisting[caption=#2,label=#1]{#1.pov}
\end{itemize}
}


%----------------------------------------------------------------------------------------
%	DOCUMENT STRUCTURE COMMANDS
%	Skip this unless you know what you're doing
%----------------------------------------------------------------------------------------

% Header and footer for when a page split occurs within a problem environment
\newcommand{\enterProblemHeader}[1]{
\nobreak\extramarks{#1}{#1 continued on next page\ldots}\nobreak
\nobreak\extramarks{#1 (continued)}{#1 continued on next page\ldots}\nobreak
}

% Header and footer for when a page split occurs between problem environments
\newcommand{\exitProblemHeader}[1]{
\nobreak\extramarks{#1 (continued)}{#1 continued on next page\ldots}\nobreak
\nobreak\extramarks{#1}{}\nobreak
}

\setcounter{secnumdepth}{0} % Removes default section numbers
\newcounter{homeworkProblemCounter} % Creates a counter to keep track of the number of problems

\newcommand{\homeworkProblemName}{}
\newenvironment{homeworkProblem}[1][Procedure \arabic{homeworkProblemCounter}]{ % Makes a new environment called homeworkProblem which takes 1 argument (custom name) but the default is "Problem #"
\stepcounter{homeworkProblemCounter} % Increase counter for number of problems
\renewcommand{\homeworkProblemName}{#1} % Assign \homeworkProblemName the name of the problem
\section{\homeworkProblemName} % Make a section in the document with the custom problem count
\enterProblemHeader{\homeworkProblemName} % Header and footer within the environment
}{
\exitProblemHeader{\homeworkProblemName} % Header and footer after the environment
}

\newcommand{\problemAnswer}[1]{ % Defines the problem answer command with the content as the only argument
\noindent\framebox[\columnwidth][c]{\begin{minipage}{0.98\columnwidth}#1\end{minipage}} % Makes the box around the problem answer and puts the content inside
}

\newcommand{\homeworkSectionName}{}
\newenvironment{homeworkSection}[1]{ % New environment for sections within homework problems, takes 1 argument - the name of the section
\renewcommand{\homeworkSectionName}{#1} % Assign \homeworkSectionName to the name of the section from the environment argument
\subsection{\homeworkSectionName} % Make a subsection with the custom name of the subsection
\enterProblemHeader{\homeworkProblemName\ [\homeworkSectionName]} % Header and footer within the environment
}{
\enterProblemHeader{\homeworkProblemName} % Header and footer after the environment
}

%----------------------------------------------------------------------------------------
%	NAME AND CLASS SECTION
%----------------------------------------------------------------------------------------

\newcommand{\hmwkTitle}{POVray Imaging} % Assignment title
\newcommand{\hmwkDate}{June\ 30,\ 2015} % Date
\newcommand{\hmwkClass}{Procedure} % Course/class
%\newcommand{\hmwkClassTime}{10:30am} % Class/lecture time
%\newcommand{\hmwkClassInstructor}{Jones} % Teacher/lecturer
\newcommand{\hmwkAuthorName}{Hyun Joo Hwang} % Your name

%----------------------------------------------------------------------------------------
%	TITLE PAGE
%----------------------------------------------------------------------------------------

\title{
\vspace{2in}
\textmd{\textbf{\hmwkClass:\ \hmwkTitle}}\\
\normalsize\vspace{0.1in}\small{\hmwkDate}\\
%\vspace{0.1in}\large{\textit{\hmwkClassInstructor\ \hmwkClassTime}}
\vspace{3in}
}

\author{\textbf{\hmwkAuthorName}}
\date{} % Insert date here if you want it to appear below your name

%----------------------------------------------------------------------------------------

\begin{document}

\maketitle

%----------------------------------------------------------------------------------------
%	TABLE OF CONTENTS
%----------------------------------------------------------------------------------------

%\setcounter{tocdepth}{1} % Uncomment this line if you don't want subsections listed in the ToC

\newpage
\tableofcontents
\newpage

%----------------------------------------------------------------------------------------
%	PROBLEM 1
%----------------------------------------------------------------------------------------

% To have just one problem per page, simply put a \clearpage after each problem

\begin{homeworkProblem}[Procedure 1: Unpacking coordinates to generate .pov file]
1. \textbf{unpackpov.cpp} takes an input file (manual) that has the coordinate and diameter information of all N particles and sorts the information into 6 separate files, each containing the x/y/z/diameter/\# particles*27 (for pbc images)/box vol information of the particles.\\
\begin{enumerate} 
	\item When \textbf{print\_double\_c} boolean is set to true ($\neq$ 0), will only print particles that are making 1+ contacts (unphysical situation)
	\item When \textbf{ignorepoi} is set to true ($\neq$ 0), will ignore poi.dat (generated by neighborchange.cpp) and will consider \underline{all} particles.\\
	When \textbf{ignorepoi = 0}, will read in poi.dat and only consider the particle index documented in poi.dat\\
\end{enumerate}

2. \textbf{bubblescutoff.py} takes the output files from 1 as input files and generates a .pov file\\
This .pov file has all the sphere coordinates and radius information. Post-processing is necessary to adjust light settings, camera settings, sphere colors, and carrying out various CSG operations.\\
Post-processing details will be provided in \textbf{Procedure 2}.


%\perlscript{homework_example}{Sample Perl Script With Highlighting}
%\cscript{unpackpov}{unpackpov.cpp}
%\pyscript{bubblescutoff}{bubblescutoff.py}


\end{homeworkProblem}

%----------------------------------------------------------------------------------------
%	PROBLEM 2
%----------------------------------------------------------------------------------------

\begin{homeworkProblem}[Procedure 2: Post-processing .pov file]
\textbf{Post-processing the .pov file generated in procedure 1}.\\
\\
\underline{Some basic POVray commands:}\\

\begin{enumerate}
	\item POVray's Coordinate system is a \textbf{left-handed} coordinate system. XY is your typical left hand open, the back of your hand facing your face, and the Z axis is the direction your left hand rotates inwards.\\
	\item Standard include files: "\#include colors.inc", "textures.inc".. etc\\
	\item Objects:
		\begin {itemize}
			\item Sphere: \textless x, y, z\textgreater \hspace{1px}, \# : coordinates of center of sphere with radius of \#
			\item Plane: \textless x, y, z\textgreater \hspace{1px}, \# : vector perpendicular to plane specifies the location of the plane \# away from the origin along the vector
			\item Box: \textless x, y, z\textgreater, \textless x, y, z\textgreater : near lower left corner and far upper right corner
		\end{itemize}
	\item Texture added to objects:
		\begin{itemize}
			\item texture \{ pigment \{ color specification \} \}
				\begin {itemize}
					 \item color specs: 
					 	\begin {itemize}
							\item rgbt \textless \#, \#, \#, \# \textgreater \hspace{1px} rgbf \textless \#, \#, \#, \# \textgreater: rgb - standard red/green/blue indicator. t/f - transmit/filter
							\item simple names: "white", "yellow".. etc
						\end {itemize}
				\end {itemize}
			\item texture \{ finish \{ texture specification \} \}
				\begin {itemize}
					\item ambient: "light that is scattered everywhere in the room" - \# indicates light that is added to texture of everything else
					\item diffuse: controls how much of the light comes directly from any light sources
					\item reflection: mirror-like reflection
					\item phong : 3D lighting ((causes bright shiny spots on the object that are the color of the light source being reflected.))
				\end {itemize}
		\end{itemize}
	\item CSG: Constructive Solid Geometry - \textbf{\textit{Can also be specified by texture options}}
		\begin {itemize}
			\item Union: two or more shapes are added together
			\item Intersection: two or more shapes are combined to make a new shape that consists of the area common to both shapes
			\item Difference: initial shape has all subsequent shapes subtracted from it
			\item Merge: like union, except surfaces inside the union are removed (gives transparent objects)
		\end {itemize}
\end{enumerate}

%\povscript{bubblescutoff3_master_trunc}{final processed .pov file}


%\problemAnswer{
%\begin{center}
%\includegraphics[width=0.75\columnwidth]{example_figure} % Example image
%\end{center}

%\lipsum[3-5]
%}
\end{homeworkProblem}

%----------------------------------------------------------------------------------------

\end{document}